\chapterwithauthor{Sanjiban Bairagya}{The motivation behind contributing to KDE}

\authorbio{Sanjiban Bairagya studied Information Technology at the National Institute of Technology, Durgapur, India. He currently works as a Software Engineer at Magicpin. He made his first contribution to KDE in 2013 and did a Google Summer of Code project with KDE's Marble team in 2014.}

\noindent{}About 20 years ago, Matthias Ettrich proposed the creation of an easy-to-use desktop environment in which users could expect things to look, feel, and work consistently, something which was lacking in the Unix desktop at the time. This was the initial motivation that led to the birth of the KDE project, after a lot of people started showing interest in the idea. Initially the K in KDE was supposed to be termed as “Kool”, but very soon it was decided that K should not stand for anything in particular, just call it “K Desktop Environment”. I was introduced to KDE during my second year of college by some of my seniors around 4 years ago in 2012. That's when I installed Fedora with KDE, as my first Linux-based setup. At the time I looked at it as just a cool desktop having so many fun applications and widgets to use and play around with. Never had I given a thought in the beginning that I could be a part in building\todo{native speaker check} these applications as well. It was only towards the beginning of the following year that I thought of starting to contribute to its applications by coding. I used to use Marble, a virtual globe software, more than the others, so I started with that. At the time Marble had a pretty basic UI, with some new features just introduced that year, like the stars showing realistic colors and constellations, and draggable panels replacing the tab-based controls. The git reviewboard was the place to upload your patches at the time. And overall KDE was focused almost entirely on creating desktop applications. But more than the applications, its the welcoming community that encouraged me to keep continuing. The first time I had pinged Dennis, one of the main Marble developers, about contributing to Marble, in the \#marble channel was in Jan 2013. If the conversation wouldn't have proceeded in such a welcoming way in which it did, I don't think I would have been encouraged to go any further. But that was just the beginning. And KDE has really evolved a lot since then.

KDE has gradually evolved itself from being a desktop environment, to now being part of almost every aspect of life that can be touched in terms of technology. The KDE Community has apps on Android now as well - KDE Connect and Marble Maps to name a few. KDE via Plasma Mobile even offers a free platform for mobile devices today, with a prototype already available. KDE Frameworks has come a long way, with its recent KF5 release. We even have a new place to upload our patches for review now, Phabricator, making it much easier to track their progress.  Not only in terms of technology, the community has also become more diverse and vibrant now with more people contributing and participating in sprints and conferences from all across the globe. Not to mention, we have united our vision as well: “A world in which everyone has control over their digital life and enjoys freedom and privacy”. And that is true indeed. It has been 4 years already that I have been using KDE software. Mostly because it undoubtedly does produce the best and most user-friendly products that free software has to offer. In fact, I grew so fond of it when I was in college, that we had even organised an event titled “Contributing to KDE” in the seminar room there, which was a grand success as well. And it did not stop there. Even now in the office, I installed Kubuntu on my laptop and after seeing me, some of my colleagues have switched to using Plasma and other KDE software as well. As a user and developer of KDE software, I feel it is my duty to spread the information as much as possible, and I have been doing just that. KDE gave me wings. The wings to know the world so much more, and be able to spend time in working with the most humble, knowledgeable, and welcoming people that can exist on planet Earth. 

Now, looking at what the future of the KDE community might turn out to be in a few more years from now, I think we should already acknowledge the fact that it is already the second largest Free Software community in the world right now (behind the Linux kernel community), and its numbers speaks for itself about the amount of freedom and friendliness that exists in the community. The rate at which KDE has progressed that I have seen in the last 3 years, there is no doubt but to know that there is no downhill from here. Having a conversation with the folks at KDE, you feel more closer to home than anywhere else, no matter which part of the world you come from. KDE has been among the leaders in desktop, and soon, with more people spending time on their phones than their desktops, it has entered the mobile platform as well. With the release of Plasma Mobile in full, people should be able to see a new world of KDE in the future as it keeps on developing. It's been 20 years since KDE was born, and if I imagine how it might be 20 years ahead from now, I can only see unparalleled progress. What could come next? Artificial Intelligence? Virtual Reality? Well, with its current rate of progress, along with its drive to fulfill its vision of everyone having control over their digital life, who knows, maybe it is evident that KDE will definitely keep spreading to the newest and latest technologies that will become a part of life for everyone in the future. As for us developers, what drives us to keep contributing is the very realisation that our code touches so many lives and makes a difference. Coming back from a tiring day at the office, I think that contributing to free software you love to use is a very productive way to spend your free time. It can be any free software, I choose KDE because of its community of people with its free and open culture.
