\chapterwithauthor{Kévin Ottens}{Meet the Gearheads}

\authorbio{Kévin Ottens has more than 12 years experience working with Qt. He is
also a long term contributor of the KDE community where he focused for a long
time on libraries API design and architecture. Graduating in 2007, he has a PhD
in artificial intelligence. In 2012, he participated in the creation of the KDE
Manifesto. Nowadays he spends time rethinking his job via a strong interest in
Software Craftsmanship. Kévin's job at KDAB leads him to contribute to Qt~3D but
also includes giving trainings and the responsibility of community liaison with
KDE. He still lives in Toulouse where he serves as part time teacher in his
former university.}

\section*{Introduction}
As I am writing these lines (in 2016), I am 36 years old. I have been a user of KDE
products since 1997. It has been 19 years, almost half of my life, almost
as long as the community existence. I finally decided to get involved with KDE
as a contributor in 2003, almost a third of my life. Also, I intend to stay
involved somehow. If everything goes well (for the community and also for me),
at one point my life without KDE will be anecdotical compared to my life with
it.

Why am I telling you all this you may ask? Well, I just want you, dear readers,
to realize that it has been an awfully long time! And now you might just wonder
\emph{"Why? Oh Why? Someone in his right mind would do something like this?
Why spend your youth working on something which won't make you rich or famous?
Why even contemplating doing that while being older? Don't you have a life?"} \\

These are a lot of valid questions and I'll try to answer it simply with this:
\emph{you can't and shouldn't get rid of your family}. That might still sound
mysterious, so let's expand on this idea.

\section*{How It All Started}
For as long as I can remember, I wanted to do something related to computers and
later decided that I'd at least try myself at Artificial Intelligence topics because of William Gibson's
Neuromancer. In 1996, I bumped into my first Linux distribution (a Slackware
spin) and got hooked on the Free Software culture as a result. It was only a
matter of time before I ended up contributing to something. \\

I had this hunch that I would likely contribute to something I use and so I
tried to use mostly Free Software with source code I felt comfortable with.
Yes, it means I chose by looking at the code of the software I evaluated, not
by its looks or by how many features it stacked (those criteria came only
distant second). Needless to say, that quite a lot of the software at the time
wasn't really to my liking. In the end, the source code coming from KDE
impressed me the most, so I started to use more and more of it, waiting for an
opportunity.

Finally, I found a missing feature for my workflow in the desktop of the time
and decided to make a plugin for it. Instead of keeping it for me, I decided to
contribute it to the world and was greeted by a crazy Canadian who maintained
Kicker at the time. My plugin ended up part of the next official release. \\

That was obviously a very encouraging start. Still, the story thusfar has been  mostly about
technical facts and some distant political vision. Nothing which could explain
being engaged for long with the same community.

\section*{"Happy Families" Meets the Lunatics}
I started for technical reasons, but really, I have stayed because of the people.
When I first got into KDE, it sometimes acted like a large, funny and chaotic family.
It was also quite dysfunctional at times... like most large families. I think it
still is like that. This is probably important for bonding with people. Obviously,
you see quite a few stereotypes in such families. I think I witnessed quite a
few of them. \\

Indeed, while getting in the community I met plenty of characters... \\

The clever emo-goth cousin wearing only leather and weird boots. You don't
always feel comfortable with him since you're not sure how he is going to react
when you might point out something flawed he did. He also tends to express
weird and offensive opinions in public just for the sake of it.

The aunt you admire hoping to be one day a bit like her. The clearly inspiring
person that takes you under her wing with great pleasure. If you pay attention
you will likely get insights from her wisdom. You might not understand it at
the time to only realize its significance later.

The uncle you like to tag along with because you know the time spent will be
fun and crazy. He gets you to places you'd never go to (karaoke bars anyone?)
and do crazy things, like talking to drunken strangers in a foreign town.

The grandpa who is a great story teller. You always love spending time with him
late during family reunions. When the great noisy time is over, he is still
around gently telling real (or made up) stories.

The grandma who has lived several lives. Clearly you can gain wisdom from her
experiences. That said, you also get the feeling that sometimes she is getting
overly conservative and over-protective. \\

As a long timer, I also ended up in the oldies group, which gives you a whole
new perspective... \\

Two sisters you appreciated, who had stopped talking to each other because of some feud.
There might have been some valid reason years ago. Unfortunately, years after
instead of being put to rest, one is estranged and some relationships in the
family are still tainted by it.

The little cousin with attention deficit disorder who brings plenty of new activity
ideas for the whole family, but rarely delivers completely
and someone else picks up to complete his vision.

The little brother still trying to find his place in the family. He is still
feeling insecure, but in time he'll grab the torch and move the family forward
together with his whole generation. It is a privilege to see him grow. \\

Of course, there are many more such characters in our community and that is what
makes it interesting and unfortunately I can't list them all here. My hope is that
by pointing a few out bluntly it'll help members to reflect on their peers and
try to improve how the family works so that it is rarely dysfunctional.

\section*{Conclusion}
Yes, I do have a life and it involves all of the KDE family characters in some
twisted way. This is not the kind of thing you really think about as a child. You
don't envision something quite like it. But it happens. Most of us start because
of some itch to scratch or technical curiosity, this is hardly a rational choice
in the first place. \\

Similarly, I'll slightly change Desmond Tutu's words: "You don’t choose your
family. They are [the Universe's] gift to you, as you are to them." \\

Clearly, KDE has been a gift to me, a second family. This is why, just like for
my first family, I try to be available for fellow gearheads and be a gift to
them.

