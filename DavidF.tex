\chapterwithauthor{David Faure}{Did you know?}

\authorbio{David Faure joined KDE in 1998, initially for a dialog box for a `talk` daemon, then contributing to many different parts of the software: file manager, web browser, office suite, email client, core technologies, Qt, and anything else where a bug prevents him from being productive. He works as a software engineer, consultant and trainer at KDAB.}

\noindent{}As one of the members of the community who has been active for the longest time, I sometimes see it as my role to transmit the memories and knowledge of the early days to the more recent generation of KDE contributors.

Did you know, that "KDE" initially stood for "Kool Desktop Environment", but the K quickly lost its meaning, because Kool wasn't so cool after all?
One of the very early contributors, Kalle Dalheimer, sometimes joked that it meant the "Kalle Dalheimer Environment", of course :-)

Did you know, that choosing Qt as the base technology was just to get started, with the idea that it was really just the GUI bits, and that it would always be an option to switch to another toolkit at some point? Yeah right :)

Did you know, that the very beginning was a window manager (kwm), a panel (kpanel), and the first library classes where KConfig (whose format was inspired by win.ini) and KApplication, as a place where to instanciate the KConfig class? It took us 16 years to get rid of KApplication again (to give KF5 a more modular design).

Then comes in my hero: Torben Weis. His creativity was amazing. He was not afraid to start huge projects, with excellent architecture and design skills. He started with the filemanager kfm, which became Konqueror and also gave birth to KIO. Then he thought "how about I start an office suite?". This is where I came in, to take over maintainership of kfm while he would go and start KOffice (known today as Calligra).
He was also a very fast coder, so people assumed he had clones of himself helping him. An old joke in the community is "Torben broke the cloning machine", which explains why we weren't able to benefit from the same trick ourselves.
He unfortunately left the project long ago (my interpretation is that he was more into initial designs than long-term maintenance), but his influence on what KDE is today is still very important, and not just because `kioclient5` shows file:/home/weis/data/test.html in an example :-)

What did we use before git? Subversion. And before that? CVS. And before that? Just an FTP server where people would upload their sources!
When I joined CVS was already being used. Oh the memories of that archaic version control system...
\begin{itemize}
 \item Even `cvs diff file.cpp` requires a connexion to the server, not convenient in an era of 56k modems (I couldn't be connected all day at home, only at university).
 \item You want to rename a file? You need to ask Coolo (which is Stephan Kulow's nickname).
 \item On the other hand you could check out a single file from a branch, we lost that feature. OK, I wouldn't say I miss it :)
 \item Talking about branches, how do you call your feature branch where you rework e.g. KIO's architecture? "make\_it\_cool" was a very often used branch name. There was definitely something about the word "cool" in those days...
\end{itemize}

You might have heard that we used CORBA at some point in the past (for component embedding using inter-process communication). Did you know that there were no stable releases with it, only KDE 2.0 alphas and betas? One of them was even codenamed Krash. This experiment (with a CORBA implementation called Mico) didn't prove successful (the worst part that I remember was that we ended up duplicating all of the menu and toolbar APIs as remote procedure calls) so we had to find a replacement, see below.

Did you know, that we had developer meetings before they were called Akademy? I recently saw a discussion where someone said "DCOP was invented at Akademy long ago". This is not exactly right. Before Akademy (which is not just a name, it also means a conference with presentations the first days) we had developer meetings (nowadays known as sprints). And indeed DCOP was invented at one of these. 
The story is that Matthias Ettrich and Preston Brown got drunk and declared that they would write a CORBA replacement in one night. The result is DCOP (later on replaced with D-Bus, for interoperability with other freedesktop projects, but I'm pretty sure that D-Bus took inspiration from DCOP, so it's an evolution).

Did you know that Trysil was a ski station in Norway? More interestingly it was the location for two KDE developer meetings, the first one just after the switch from CORBA to DCOP.
The daily cycle at that event was interesting... Summer in northern Norway means that the sun never really goes down. At 4am the sun becomes slightly less visible for half an hour, and when it comes back up again we would then realize, OK, it's time to go to bed. We'd sleep until noon, have a copious breakfast (well, brunch), hack until dinner, then hack again until 4:30am. Unusual but effective - we got lots done. The hungry mosquitos are another story though...

Torben joked that he was a bit ashamed of the bad design decision to go with CORBA (and out-of-process component embedding in general), so his next project, with in-process components (which became KParts) was initially called Canossa. Did you know that Canossa is a village in Italy? OK but what's the relation? In January 1077 the Holy Roman emperor Henry IV did penance at the castle in Canossa to obtain a pardon from his excommunication by Pope Gregory VII. We quickly switched to the KParts name though :)

The second Trysil meeting, much later (in 2006, before KDE 4.0), is also a source of very good memories. The contributors coming from America had a bit of a jetlag, so George Staikos was sleeping, sitting on his chair, while his code was compiling, then waking up again to keep hacking/testing. And while Celeste was sleeping on the sofa, other people put a "do not disturb" sign on a sticky note on her head. Meanwhile Coolo was drawing German flags on the forehead and cheeks of anyone who wouldn't run fast enough, due to the Euro Football Championship happening at that time.

All of this, and much more, constitutes the history and culture of our community, and I'm happy to have this opportunity to keep passing it along to more recent contributors. The creativity and enthusiasm of the early contributors is still something we can draw inspiration from today.
