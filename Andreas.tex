\chapterwithauthor{Andreas Cord-Landwehr}{A New Generation} 
\authorbio{Andreas Cord-Landwehr joined KDE in 2011 and primarly contributed to KDE's educational software applications. After his PhD at Paderborn University in the field of algorthmic game theory in spring 2016, he joined CLAAS E-Systems as a software developer.}

\todo[inline]{still needs first review}

\noindent{}When my aunt tells me about her youth, about using horses and carriages to harvest the hay at my grandparent’s farm, it sounds like a story of long ago. When I tell some high school student about twenty years ago, about going to the library to borrow programming books, since we did not have Internet at home at that time, it must sound like a story similar ancient like my aunt’s one. As hard as it is for me to imagine a world where we still use horses for our daily work, it might be similar hard for someone of the ``digital native'' generation to understand the life in the world just twenty years ago – albeit it is just one generation.

Twenty years ago, it has been the time of the struggle out of the Microsoft vendor lock-in. At that time everyone I knew used a Windows PC, even me. Actually, my KDE and Linux memories only start some years later. It must have been about fifteen years ago from today, when I got my first Linux Boot CD and got excited by a booting screen that really showed me what happens on the system instead of a Windows screen that liked to magically freeze during boot. Still, I can remember my first log-in into the KDE desktop, my first bug report, my first system configurations. -- It has been an exciting time. Updating the KDE desktop always brought tons of new features, new applications that I was missing from the Windows world, and I really felt the freedom and adventures this new world gave me. For me, as someone who joined the ranks of the KDE developers several years later, I can only guess how exciting the times must have been in particularly as a developer back then.

Today, the world is a different one. Linux is an important ecosystem and deeply rooted in the economy. The KDE desktop together with the Linux ecosystem provides applications for all needs and it is a rare event when one encounters a missing application from the Windows world; actually, sometimes at work I have it the other way around. However, the trend of the desktop as the predominant computing system is changing. In economy, the desktop will surely remain and still will be an important tool for many decades to come. But for our daily lives it is already different. The desktop computer is slowly becoming an office tool, like a typewriter in old times, a thing that you do not place in your living room but rather in an office. Already today, the desktop computer is not the first device in your hand if you want to look up some information. Some years ago it was very different.

The question is, what does this mean for us as the KDE community, a community that started around the goal of making a great desktop for end users. I believe that this new trend is both a challenge and an opportunity at once. We have our main products, which target at the desktop, a solid platform which will be used still for a long future. But there are the new fields, the smart phones, the tablets, the watches, and the countless devices that try to make your lives easier. Compared to our starting grounds the signs for these new fields are not bad, actually free software found its way into many parts of these devices. Now we have to identify what the challenges are that we want to address in the future. With the KDE Vision in place we are in a good position for doing this.

One major challenger will be to find a good balance of what to preserve and what to start new. We have a long tradition as an open source project but should not become a dinosaur waiting for a comet. We are active in hot topics, but should not become a flock of lemmings running mindless in one direction. In a community we are people of various backgrounds, educations, and ages and with twenty years completed, we finished what one can call a generation. I wonder, how does someone perceive this world who is young enough that they cannot even remember KDE 4.0, a world that always had the Internet, where the first access to a computer was wiping a hand. With each generation the perception of the world and of the way we deal with it is changing. This even changes the fights we see worth fighting. As drastic as the change from horses to harvesters was, I see the change from the open source world twenty years ago, with the KDE desktop still being an infant, to today. But since not only the tools and techniques are changing, we as a community must evolve to the next level, to update the purpose that drives us. Twenty years are one human generation, I am curious what will be after another one.

