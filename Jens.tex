\chapterwithauthor{Jens Reuterberg}{Say yes} 

\authorbio{Jens Reuterberg is an illustrator and designer (but always illustrator first) living in Sweden with husband, cat and laptop. Since 2012 he has used Open Source for work exclusively, and since 2014 as his work exclusively since he started KDE's Visual Design Group. Sources have it he even talks in his sleep.}

\noindent{}There are things in life that shape you. The little things that make your life take a sharp turn and wheer off in an entirely different direction.
They happen constantly, it's how you met your partner, it's how you got the new job, it's how you got started playing the banjo. Often they are tiny little incidents of no remarkable relevance in and of themselves. A man slips on the concrete, you laugh and someone else laughs - you start to talk while in the background he brushes himself off and walk away and five nights later you wake up next to them and wonder how you could have even considered life “living” before.

We take them for granted, the little strange happenings that you can't foresee. Like they where a natural effect to you being you. How could you NOT have started playing the banjo, you might ask yourself. If it wasn't for that time you by mistake ordered one online you would probably have picked one up at some point, wouldn't you?
It is amazing that, with the billions of us living here on a huge mass of land, the small chance that any one of us would meet any specific one of everyone else of us, is never really remarked upon with greater fascination than it currently is. Like it “just happened”.  Or how from an impossibly large selection of options you just chose the right one, at the right time, while being in the right frame of mind. It just happened.

In my opinion - the method with which to gain the greatest amount of “just happenings” is to allow for the little off-chance things to happen, to the greatest extent of its capacity. Let go. Say yes. Go “oh just this once” once more. Accept that you will look like an arse no matter what you do, so you might do it anyway and in a way you yourself could foresee and somewhat control than by just the act of being you every day. 

That is why I am in KDE now. My willingness to look like a complete and utter arse. My readiness to say “yes” to things I am not totally sure about.

Obviously this is not the only reason. Other people's capacity to accept me even though I look like an arse are just as relevant as my own ability in being one of course.\todo{native speaker check} Others' willingness to ask, is as important as my ability to say “yes”. But that is, simplistically, the reason why I am in KDE at any rate. Saying “yes”. Looking like an arse. But mostly saying “yes”.

I was asked by Sebastian Kûgler to show up at the Plasma Sprint in Barcelona. He in turn had had me suggested to him by Aaron Seigo and those two, my “yes” and ability and acceptance of risking to look like arse was why I was standing outside of Barcelona airport alone.

Now I am a fairly awkward person. I know this. I can be either painfully silent and withdrawn or overly jovial and with the voice and body language of an opera singer for the hard of hearing having a stroke. Standing outside an airport, about to go to an office filled with strange programmers, a profession I don't grasp at all, about to do “design-stuff” for a week in an area I had never worked in before in my life, Open Source - I was in “silent-mode”. 
At that point, catching a smoke between flight and bus ride, I regretted it greatly. I wanted nothing more than to go back home. To be fair I had spent the entire flight from Sweden regretting it. The bus ride from my home to the airport outside of Gothenburg had been a clear and constant battle to tell the bus driver to stop because I'd gotten on the wrong bus and had to walk back home through the woods. The act of going out the door was the emotional equivalence of a nuclear bomb strike of regret - regretting I had ever said “yes” when Sebastian asked me.

But it is those slapdash and shot-from-the-hip yeses that do it and that understanding was what had made me say it in the first place. Saying a skeptical “no” makes things easier but it's the yeses that deliver the potential for “just happens”.

Not that I cared then for the bravery of my past. It's so easy to be brave before the battle. So simple to be self-confident before the test. To say “yes” then when you now want to scream “no”.

I spent the bus ride from the airport into town in abject terror and was dumped at Plaça d'Espanya with suitcase, hand-drawn map and confusing self-doubt about the whole thing.
What was I meant to do there? I had this vision of me, sitting alone in a corner checking Facebook and pretending to laugh at C++ jokes while work went on around me. Of questions with answers I could hardly spell - let alone deliver - and a group of programmers lying about having to all go to the loo in another area of town at the same time just so they could have beers and ask each other “why this Swedish mustaschioed dickhead is here”. If optimism and hope for the future is a blue bird swooping through the sky, I was a bright orange elephant trying to maneuver a burning 747.

Now had I, at that time been able to see just a few days into the future - had I been able to, from that chance online encounter, question and temporarily insane “yes” on my part, extrapolate what would come, I would have been happy as nothing to be there. 
You see, KDE is a mess of people. Tons of us and all going in [Number of KDE people]+1 different directions at the same time. We stretch a rather broad gamut of humans, from the loud to the quiet, the kind to the apathetic, the professionals and the hobbyists, the productive to the lazy. “Anyone can join this club” could have been an insult, but for KDE it is more like a battle cry of clear intent.
Had I known then, that the week that followed was not just productive and thought provoking but fun, I wouldn't have been nervous at all. Had I known I would meet people who I now consider close friends, who have that rather adorable quality of being incredibly intelligent, but seemingly not able to grasp how much more intelligent they actually are, treating some random Swedish arse as an equal instead as a befuddled moron - well I would be cartwheeling all the way through Barcelona to get to the sprint.
In the days to come I talked technically complex issues with people who can do what is best summarized as “magic” as far as I’m concerned, people who didn’t talk down to me, who didn’t ignore me but instead explained every issue in a way that I could take part. Who listened to my ideas and more importantly explained why some of them would not work and some might.
I talked about design with people who listened to me, asked questions and suggested ideas with the casual ease of a giraffe eating a trimmed hedge. People who inspired, not just amazing work in me but amazing work with them.
Had I known, at that time, on that street, that I would go home later that week, fall asleep next to my husband and dream about interaction design, UX and visual elements and wake up the next morning pouring all energy into this “Open Source” thing - I would probably not have believed myself anyway, but if I had I wouldn't have been the least bit worried or scared of having said that initial “yes”.

But I hadn't so I was, as I stumbled along that street through Barcelona with my map, backpack and social anxieties in tow.

I had drawn the front door of the Blue Systems office by hand on a piece of paper from Google Streetview simply because at the time that had felt more calming than taking a screenshot and the little drawing I kept holding up to compare different doors to was getting damp and soggy from my sweaty hands. When I found the door, that was also the moment I met my first KDE developer: Ivan.

I don't know what clued me off to him being one of the developers. It could have been the bags since he had also come from the airport or maybe it's the nerds innate ability to recognize a sibling but I asked “You here for the Plasma Sprint?” 
He looked at me as if he had yet to deduce if I was there to murder him, or just rob the office. As if some kind of debate on whether to pretend not to speak English and walk away or reply was raging inside him. 

After a rather too long second or five he seemed to make up his mind, take his chances and go “yeeees...?”, pressed the intercom to the front door, opened it and let me in.
