\chapterwithauthor{Albert Vaca}{Staying Relevant} 

\authorbio{Albert Vaca is a software developer from sunny Barcelona, with experience in Android and C++. He maintains the KDE Connect app for Android and PC.}

\noindent{}When I was first asked to talk about KDE's history, I thought that I am too new in the community to have something interesting to say. My first contact with KDE (as in "the KDE desktop") was in 2006, and I didn't touch any code until 2010. However, maybe with some luck, my newcomer experience might bring a different point of view to the whole picture.

The first thing I want to note is that I first approached Linux and the free software world in a time when it was booming. The market share of the Linux desktop was not huge, but free software was trendy: schools and public administrations where adopting it (even if only to save costs), it was in the news, a lot of innovation in browsers, and security and other areas came from open source projects. For all of these reasons, it was a time when a lot of people, who like me were interested in geeky-computery-stuff, got into the free software world.

Many have said that one of the reasons for this boom was the failure of Windows Vista. And yes, Vista was not a success, but in hindsight it was probably the least of the problems Microsoft would have to face. The major game changer, in my opinion, came a bit later: the popularization of the smartphone. And there, the Linux desktop was hit in the same way Microsoft Windows was.

The reality is that, after the smartphone revolution, computers are not that relevant anymore. Therefore, in my opinion, the less relevant the desktop is, the less relevant we are as a community: most people get into the KDE Community through our desktop environment and desktop apps. 

This means, to start with, that we will not see that many geeky teenagers like me coming to us. This is perfectly normal: people today are more interested in context-aware, notification-enabled Android and iPhone apps, than in big bulky desktop suites that you manually launch to perform a task. Sadly, there is not much free software available on these new platforms.

Our current situation is that most people in KDE are people who work on and know about desktop software. Of course, most of these people are not going to stop developing for the desktop just because it is not trendy anymore. Instead, what I would like to see happen is engagement with a new generation of developers who have the ability to grow within KDE a new family of products relevant for them and the way they use the technology. That is, a generation of developers who understand what needs to be done in order to reach the users of the emerging platforms, and who want to do it from within free software.

Only by achieving this will we manage to reach a whole new generation of people and get them interested in using (and maybe eventually developing) free software, and to continue to grow our community. Non-free software is ahead of us, but the same way we did it in the Windows monopoly era, I am sure we will put our focus again on providing software which fits the users’ needs better. On these new platforms, there are plenty of new areas where free software can excel above non-free. Probably the most important one: the privacy of users.

At the same time, of course, we still need to maintain our position as one of the best desktop free software communities in the world. Here, though, we can learn a lot from the emerging platforms and adopt what they did well. Just to name a few things: sandboxed apps with discrete permissions, great focus on the user experience, standardized distribution mechanisms from developers to users, context-aware apps, and more. Non-free desktop platforms are also trying to do the same (an example of this are the dramatic UI changes we have seen in each version of Windows from 7 to 10), so we have the opportunity to have a big impact by being, one more time, faster than them.

In conclusion, even though the personal computer is not going to die anytime soon, things are changing fast and we need to keep up with the good work we have been doing until now. At the same time, though, I think we need to find a way to reach all the people who don't use traditional desktop software, but who we believe would benefit from using free software.

In the end, we need to understand what we use technology for and how we can make sure that it serves our society now and in the future. We all agree that free software is the way to achieve this goal, so we have to make sure it is present in every front where technology is involved. It is then when we will be the platform that makes possible the society we believe in.\todo{native speaker check}
