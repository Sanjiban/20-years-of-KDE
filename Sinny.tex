\chapterwithauthor{Sinny Kumari}{A place to stay forever}

\authorbio{Sinny Kumari joined the KDE family during late 2010. She worked mainly on Plasma Media Center and parts of Plasma. She loves talking about the KDE Community and its software. Other than that, she works toward motivating others to work on Open Source projects. In the daytime, she is a Software Engineer at Red Hat where she helps in maintaining Fedora and RHEL for PowerPC and s390x architectures.}

\noindent{}It was late 2010 when I slowly started getting familiar with a new operating system called Linux which had the KDE Desktop environment with it. First, I started as a user and then slowly pushed myself to fix bugs. The first time I interacted with the community was by sending a patch for the File Watcher widget. As a beginner working on fixing the patch was not very easy. It required me to first download the source code and build Plasma and all required dependencies. Later, I had to find the right place where the fix should go and then re-compile the affected application and see whether the changes I made are working. Even though the changes I made were really small the experience I gained was not. It was an amazing feeling and experience to work with such a large community who develops loads of amazing software. I realized how different and fun it is to work on software which is used by millions of people compared to the way I was studying in college. This lead me to stick around the KDE Community even further.

My journey in KDE continued further as a volunteer for conf.kde.in 2011 in Bangalore, India. It was the first KDE conference happening in India and also the first KDE conference I attended. Thanks to Shantanu and Pradeepto who gave me a chance to be part of organizing it. It was fun to spread the news among colleges and explain to them why they should attend this conference. Lots of people came to attend it including college professors, working professionals but mainly students. I believe this conference opened the gates for a lot of students from India to contribute to KDE. For me, it was the first time I met people in-person from around the world and felt how awesome this community is. It increased my bond to KDE even further and felt like I am part of the KDE family.

With great enthusiasm, I planned to stay even longer here and contribute to KDE. As a result, I started getting into different KDE projects which interest me. I figured out that projects around Plasma (KDE’s workspace) interest me more. Meanwhile, the Google Summer of Code period was also about to start, hence I applied to Plasma Media Center - the project that has always stayed close to my heart. Plasma Media center is a KDE application which aims to provide an easy media experience (music, pictures, videos) to people on a device running KDE Plasma. It supports browsing and viewing media local to your system as well as available from online services like YouTube, Flickr etc. Many thanks to Marco Martin who mentored me to get this project in good shape. While writing this, I recall how different family members from KDE participated to shape this project even better and helped it to mature - in terms of code, design, spreading the news, packaging, motivating and so on. I remember writing blog posts and articles for KDE's official news site about Plasma Media Center releases and how happy people were to see it growing from desktop to tablets and extending to TV. KDE love is always about spreading it to more people. Later on in KDE India conferences and meetups, we used to talk about and give demos of Plasma Media Center to attendees. We explained how to create an awesome UI application like Plasma Media Center using Qt/KDE. Attendees used to love this application and liked to contribute to it. As a result, we were able to pull in even more people to work and continue evolving Plasma Media Center further along to a long journey. With continuous effort of loads of people, Plasma Media Center's codebase matured from Playground (the place where new KDE software starts its life) to Extragear (where mature software lives that is not part of the main release cycle), and then to the main release cycle. The journey of this project's success was not easy but not impossible either. It reached this maturity level only because of love from different people in the KDE Community. Sadly, these days due to job and other responsibilities I don't have time to work on it. But I am positive that it will keep getting love in the future from my other KDE family members.

Life keeps moving on further and so does love towards the KDE Community and its software. In order to maintain, improve and have new software in KDE, it is important that together we keep on expanding our family while keeping everyone involved. This friendly and loving nature of community members will always keep new/existing contributors feel welcomed and homely while working on KDE projects.
