\chapterwithauthor{Dani Gutiérrez Porset}{TITLE}

\authorbio{Dani Gutiérrez Porset is a Free Software activist. He does consulting around Free Software and teaches at the Public University of the Basque Country. He was a project manager at the Office for Free Software at the Basque Government. He organized Akademy-es in 2010 and 2013 as well as Akademy and the Qt Contributors' Summit in 2013.}

\todo[inline]{add title}
\todo[inline]{needs another proper review/editing}

\noindent{}When speaking about personal fulfillments, some of the best achievements reached by individuals and groups happen in environments where motivation is not tied to economic incomes but contribution to community, be it a big or a small one. Think about situations related to friendship, family or love. Or, beyond feelings, about an important discovery for human health.

In the subset of free software world that is formed by communities like KDE, what moves people to contribute with personal time (coding, designing, teaching, evangelizing and more) is not so much a possible chance of making business, but other factors like building a high quality product/service on your own or maybe with other colleagues, and of course sharing part of your "brain" around the world, like when you know that your Kxyz program is being used in more than, say, 50 countries.

Twenty years ago it was very hard to believe that a KDE code would be used in a place like a particle accelerator at CERN, or that a free software kernel would conquer the market of mobile devices. And these kinds of successes have been achieved not just by the contribution of hundreds, thousands of people, but also by means of a political point of view based on generosity and freedom to learn, modify and improve what past generations have been building for future generations, trying to create a better world.

Does this all mean that making money from free software is bad? Not at all. Today more and more companies are making good business thanks to open source components. Think about so many companies earning money thanks to Apache, Qt or Ruby on Rails. Free software companies have some interesting sides: it is expected that they make good products because of their "professionality", and they spread a fresh and competitive option among the traditional private market. But there is a but: in general it will be more difficult that companies open and share completely its software, because the spirit and deep motivation is mainly business\todo{needs rewording by native speaker}. And, again, it is not a bad thing (but a necessary one) that people do work for a living, or even to become rich.

Joining all of this with the field of politics, it would be great if there were a kind of basic income, publicly assured, for those people that during some part of their life dedicate themselves to produce and spread free software, in order to recognize its contribution to world.

Among distinct useful tasks related to free software communities, typically we hear some like coding, designing, documenting and perhaps lawyering. But ``free software communities'' is three words and the last one means people cooperating together.

As we belong to the IT field, the interconnnection between our people is usually done remotely, with email, messaging, social networks, or video conferences on PCs, laptops, tablets or mobile phones. But no technology comes near the direct face-to-face encounter, and it results. It is really great to meet us for the first time or even more when we have met before. It's interesting because we know each other not only from the techie side, but also regarding other important parts of our lives. And when we see and hear each other, and share a drink and laugh, and we realize our common target across the free software world, we come back home more motivated to go on working hard.

So, to set spaces where programmers, managers, designers,... smile and touch is a key for a better continuation of the community. And a big opportunity to reach this "community empowerment" are the international meetings like Akademy or FOSDEM. For this reason, for the KDE Community it is a must to prepare good Akademy encounters, mainly the technical part but also the casual and friendly one. Let's be a community where we take care of all people who belong to it.