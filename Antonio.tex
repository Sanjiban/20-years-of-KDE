\chapterwithauthor{Antonio Larrosa}{The values within} 
\authorbio{Antonio Larrosa is \todo{add bio}}

\noindent{}When someone looks at KDE there are many things that can be seen depending on the point of view and the experience of the observer. One can look at KDE and see an intuitive desktop environment and many applications tightly integrated that allow anyone to use the full power of his/her computer in an easy way. In that sense, someone can use KDE for years and only see - and benefit from - the software developed by its developers. From another point of view, if a developer looks at KDE, he or she might see not only the applications but the framework that KDE provides. With KDE Frameworks a developer can quickly write applications that use the full potential of the desktop and integrate with other applications by maintaining a homogeneous look and feel across the desktop. 

But KDE is much more than that. It's a community of people doing Free Software for real users. Let's analyze what that sentence means.

\section*{A community...}
It's a community because there's a link that bonds us together. There's a purpose in what we do: to create software that liberates people when using computers\footnote{Although there are projects to expand this to tablets and mobile phones too.}. There's a strong feeling of community in KDE. There are artists, designers, programmers, translators, users, writers and other contributors from all around the world. Literally, from all continents (ok, maybe except Antarctica). Thousands of people organized to create free software in a common direction.

Now I should point out that if you plan to contribute to KDE, my words might be misunderstood so I want to leave clear that you shouldn't let that strong feeling of community discourage you. New contributors are welcomed and indeed will feel welcomed and part of the community quite quickly. Contributing to KDE is not only good for KDE, it can help you get in touch with many really good developers, artists, translators, etc. and by getting you real life experience, it will help you build a better resume.

\section*{... of people ...}
I guess there's no need to state that, even if sometimes it looks like there are, there are no robots and no aliens collaborating at KDE. They're just people like you and me who at some point in their life decided to invest part of their time (sometimes work time, sometimes free time) in making the world a bit better through Free Software. During these 20 years of KDE existence, I've met computer scientists who collaborated with KDE, but also all kind of engineers, designers, linguists, musicians, mathematicians, physicians, teachers, retired persons and even high school students working at KDE. I remember an anecdote with respect to that from many years ago at an Akademy meeting with a developer still in high school who needed to use vectors in a game he was developing but he hadn't studied them in class yet. As a mathematician I offered to teach him what he needed for his game and he learnt some maths quite ahead of other people his age.

Since KDE contributors come from very different countries with different cultural backgrounds and customs, there's a Code of Conduct\footnote{\url{https://www.kde.org/code-of-conduct}} that offers guidance to ensure KDE participants can cooperate effectively in a positive and inspiring atmosphere.

\section*{... doing Free Software ...}
Everything developed in KDE is Free Software. It's stated as such in the KDE manifesto\footnote{\url{https://manifesto.kde.org}} so it's clear that freedom is in the core of our technology. This has many benefits over non-free software. For example, it's the only way KDE can assure that the software will be available for everyone, for all time.

Freedom is a very valuable right that is usually not perceived as worthy of our attention, until it's lost. And there's a real possibility to lose it. It's currently at risk when we use computer systems that can't be analyzed like some televisions which have a microphone always listening in on your own living room conversations and noone can check exactly what it does with everything it “hears”. It's at risk too when someone uses computer software whose license (enforced via legal or technical methods) doesn't allow the owner to share the software with some friend or family member to use in their own computer. And it's also at risk when you have some software that does something close to what you need but you can't modify it (or get someone to modify it) to do exactly what you need.

Those are only three examples, but to solve those problems, among others, Free Software guarantees 4 types of freedom:

\begin{itemize}
\item The freedom to run the program for any purpose.
\item The freedom to study how the program works, and change it to make it do what you wish.
\item The freedom to redistribute and make copies so you can help your neighbor.
\item The freedom to improve the program, and release your improvements (and modified versions in general) to the public, so that the whole community benefits.
\end{itemize}

\section*{... for real users.}

Users are the most important part of KDE. During these years I've seen people of all types use KDE, from infants to nearly 90-year-old elders, from users who were using a computer for the first time to users with many years of experience using all sorts of computers, from 10,000 PCs in elementary schools in Taiwan to 15,000 PCs in the Munich City Council. All of them use KDE with usually very positive outcomes.

But sometimes there are problems too. Usually those problems are not much different to problems that would appear in other desktops/operating systems, but with one main difference. If someone finds a problem within KDE, the very developers can be reached at bugs.kde.org where any user can file a bug report with the problem he or she found or the improvement he or she thought it would be nice to have in future versions. That way users can participate in the improvement of KDE not only for them, but for everyone else. And yes, developers are not offended when someone reports a bug in a nice, educated and useful way. On the contrary, they encourage users to report bugs since they sometimes only appear under some specific circumstances that developers can't always predict or reproduce.

\section*{Conclusion}
KDE has evolved a lot in 20 years. As you see, the work KDE currently does is important in many different ways: technically (providing an innovative state-of-the art desktop, applications and frameworks), economically (providing free of cost software for everyone to use, thus giving access to developing countries to the same technology the rest of the world uses) and philosophically (providing software that help us maintain our freedom and rights). If you also think this work is important, don't hesitate to join us and help us make the world a bit better, at kde.org.
