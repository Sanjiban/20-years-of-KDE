\chapterwithauthor{David Narváez}{A Learning Paradise}

\authorbio{David Narváez is a PhD student in Computer Science at RIT in Rochester, NY,
USA. Originally from Panama, he is the current maintainer of Kig, an application
for interactive geometry that is part of the KDE Education Project.}

\noindent{}I joined the KDE Community back in 2009 to fix a couple of bugs and
never left. I was one to scratch my own itch in any application I
would use frequently: from Kig, which was my main interest, to KDevelop
which I used to work on Kig, to Plasma because I was using it to
manage my desktop environment. KDE was certainly not the only
FOSS community I contributed code to, but no other community was as
enticing as KDE. At the beginning, I could not really point the
finger at what the reason for this was.

At the same time I was learning the trade as a professional software
engineer (which, in my culture is nothing but a name for a glorified
computer programmer) back in Panama, my home country. At the peak of
my contributions to KDE, I had a full time job as a software engineer,
coding several hours a day, only to come home and code several more
hours working on KDE-related stuff. Many would say I spent my entire
day coding, but for me these were two completely different activities.

I eventually realized that the differentiating factor was
learning. KDE was much more than a developer community for me: it was
a learning environment. My daytime job, on the other hand, did not
foster innovation and learning. This was not a problem particular to
my employer at the time, but a more general issue about the culture
among software developers in my home country. While I understand not
everybody is as excited about learning as I am, I can objectively
argue that KDE, as a learning community, prepared me better for a
global market and gave me a better chance at several other
opportunities that came by later in life. Here I describe some of the
core values I found in KDE that I could not have found inside the
software developer market in my home country:

\begin{itemize}
\item In KDE we have a horizontal structure promoting code review,
  open design discussions, collaborative coding, etc. In contrast, my
  daytime job at the time had a vertical structure where fixing bugs
  in somebody else's code was considered an offense, and trying new
  technologies was never an option during design discussions.
\item Proposing the adoption of new ideas and paradigms in my
  workplace would almost always meet the mantra of "we have always
  done it this way". In contrast, I joined the KDE Community as a
  developer just a couple of years after KDE adopted a whole new
  approach to desktop environments through KDE 4.
\item KDE is, by definition, a global community. As such, you will
  always deal with not only different time zones, but also different
  cultures. I vividly remember waking up at five in the morning to read code
  reviews from people at 7 time zones away, and improve my code based on
  their feedback before leaving for my paid work. Right around those years, the
  software development market in my home country was starting to
  globalize, and as a consequence, people started thinking about
  different time zones. When it was my turn to deal with these issues,
  it felt like home because of my KDE experience.
\item From the merely technical point of view, the various software
  architectures used across KDE projects made it possible for me to
  explore paradigms and designs that were not popular inside the
  market in Panama. My exposure to these ideas turned some of the more
  challenging tasks of my day job into straightforward adaptations of
  problems that had already been solved inside the KDE Community.
\end{itemize}

I am convinced it is because of all these values we have forged
as a community inside KDE that I was able to pursue many opportunities
that would have been way out of my reach had I been equipped only with the
knowledge I could acquire in the local market. But Panama is not the
only country in the world where the digital divide is keeping
developers with great potential from acquiring all the knowledge they
need to be competitive in the global market. In fact, most of the
skills and experience that are highly valued in the IT market (think
of fluent English, or early exposure to computers) are native to a
small fraction of the planet which we usually refer to as the first
world. In the age of information, communities like KDE play the role
of forges of global talent that will enable important changes in our
digital lives. And although my personal journey in KDE has improved my
IT skills, the global talent we attract as a community does not
necessarily need to be programmers: some of our most active
contributors work on designs, translations, technical writing and even
marketing; all of which are areas with their own global markets and
needs. What is in it for all of us, is that our participation in the
community also brings along experiences that are stepping stones in
our paths to achieve personal goals.

Today, I have moved away from the industry and into research, doing
doctoral studies. This move has meant leaving my home country,
adapting to new places and having less time to contribute code to Free
Software in general. Yet, KDE is still an important part of what I do
and, as such, I care about its future. Since I consider the KDE
Community an enabler of opportunities, I see our products as means to
a greater goal that is helping the world through innovation. This
point of view is what drives me every time I contribute to KDE,
because it is as exciting to think about where I can take KDE as it is
to think about where KDE can take us. Thus, it is my first and
foremost priority, looking into the future, to preserve this nature of
our community. I believe this, and not the technology we produce, is
the key to staying relevant for the next 20 years. In this context, outreach
and mentoring naturally translates to more opportunities as more
people get involved, and I cannot think of a better excuse to strive
for world domination :)
