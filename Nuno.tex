\chapterwithauthor{Nuno Pinheiro}{The Transient Nature of Design}

\authorbio{Nuno Pinheiro is a Portuguese graphic designer and illustrator. He specializes in iconography, themes and user interface design. Nuno's works include general illustrations, UI design, web design, corporate design as well as other works in creative areas. Known for his work in the Oxygen Project where he is the current coordinator of a design platform with 2000+ icons, wallpapers, sound effects and window styles. His computer art is used on KDE computer platforms worldwide.
Over the last years he engaged in coding with QML creating fine tuned experiences for users and became interested in the Developer/Designer Interface.
He works as a UX/UI designer, consultant and trainer at KDAB.}

\section*{The Tempos of Oxygen}
Design in open source is special. At least to me it is, and has always been, and the reason for that is simple: the relative ephemerality of it, so unlike other open source projects where the beginning is well defined but the end is something to be avoided. In design and especially in this post-modern ever-recursive design landscape the end is as present as its beginning, and so it must be fully embraced.

So... What is the point of it? I mean if it is done to have an end what does it achieve? This creates a problem that I'm sure every design related open source project has debated with itself.

For Oxygen and me personally it was solved with the assumption of three time periods: a future, present and past. That looking back has now been reverted.

\section*{Its Past was its Future}
So what drove me to design and open source? 
I guess this is where, for me, design shares more with common “traditional open source projects”: an itch to scratch, or more specifically a couple of emoticons in an instant messaging app I used in my favorite desktop at the time (KDE 3.X). So I made a couple of icons (really low quality) but was encouraged by the wonderful community to keep working on it and so I did. And as a result of my continued engagement with the community, I was invited to be a part of this new project – Oxygen. In its infancy it was a Future, it was everything, and anything, the solution to all problems. It was also a repository of all my uncertainties and doubts (all founded, I might add), but it was a fantastic time, a time to get to know what you are and how you express it, a time to realize that working within a group of people is far more challenging but also far more rewarding than all by yourself. A time to make mistakes, a time to correct those mistakes and a time to realize the extension of your errors would mean, you would need to start all over.
In the end what comes out is Oxygen, a child of its time, a son of Everaldo Coelho, Crystal Icon set of Nuvola by dear friend and Oxygen colleague David Vignoni. It was an icon set and it was amazing.
The future offspring of some superb parents - and trust me we stood on the shoulders of giants - for its time Oxygen's parents were, in many ways, ground breaking feats of computer design not just in open source but in computer design, rivaling with the best of the best in the industry.

This realization made me become fully certain of Oxygen's own future and goals: Oxygen should strive to be as successful as its parents, and most importantly, prepare the way for future offspring.

So in my mind Oxygen was something for the KDE 4.X series. Beyond that something new would have to take its place.  


\section*{Its Present, where Past and Future face each-other}
So you have this group of people, that are making an icon set, and in the constant struggle between past and future, you keep on creating new futures in order to move on, so you get more people involved in the project and you add new futures, new projects, new ideas, and the icon theme becomes so much more. This is the explosion phase. An icon theme becomes a Qt theme, a sound theme, a design platform – 1000 different things!
Personally, this is the time that made me a designer. Never underestimate the power of trial and error. A lot of practice does not make perfect but it sure helps you to get better.
The icon set suffered many mutations as it defined itself thought time, the Qt theme, Plasma themes, Oxygen and Air, the cursor theme, the sound theme, the multiple websites, the countless posters, banners, mugs, pins, meetings, talks, etc, etc,... amounted to gargantuan amounts of work. 
Make no mistake it was an absurd and gigantic effort, it was incredibly fun and in a way it set, what I personally consider, Oxygen biggest achievement.
Before Oxygen, design projects in the open source world tended to vary from uncoordinated projects that lived in the same space, to vaguely related projects where different groups would coordinate design visions so that desktops would have some sort of coherent visual language, for example: the good efforts from our friends at the GNOME Desktop, and the multitude of projects it created but that shared an obvious vision and language.
Still Oxygen was different now. It was so much more than just an icon theme. It was anything you would see in your desktop and more. We went to the extreme of making GTK themes so that the KDE experience would be even more consistent for the users. Hat tip to a great Oxygen guy, Hugo Pereira for his outstanding work in this area.
Oxygen might not have been the best icon set of all times (it was good enough in my opinion but not as ground breaking as its parents were) but the scope of design efforts was unprecedented in the open source world. 
So to me, Oxygen's development period, “present”,  was a success - maybe not from the pure creativity design language point of view (I'm not even sure if I'm unbiased to say anything truly fair about its merits in that regard), but from what it set as the goal post of what a design project in open source should be. I believe it did reach its goal by setting a new high bar in what to expect from open source design projects.

\section*{Its Future or the starting of a new Past}
Some years ago Qt released Qt 5.0 and, as anyone that knows something about KDE knows, that means big changes are coming. Add to that the mobile explosion, the touch explosion, the QML language and Qt Quick revolutionizing things and the relative importance of design in computer user interfaces.
This meant that visual languages and user expectations were changing. Also my expiration date on Oxygen was reaching its due date with series 5.X's on the horizon.

A perspective of making themes, even coordinated ones were not enough to create meaningful competitive user experiences.        
Oxygen failed to be that. As a result of the way it evolved and what it consisted of in its inception it was hard to be anything else than what it was. This new method of doing things in some ways would defeat the propose of consistent theming, just like architecture and urbanism rules are different things, so are consistent theming and perfectly tailored user experiences different and not fully compatible concepts.    
So at the end of the Oxygen period, user experience and user interface design were reaching an inflection point. Gone were the days where graphical designers challenged their own illustration skills in a perpetual "I can draw my candy more naturalisticly silly than yours".
We had reached the saturation point of the silliness in graphic representations of every day objects as user interface elements. Now back then people needed to find a culprit for it all, a quintessential word that in itself represented all evil. Cue in "skeuomorphism", a word used in traditional design to imply a faux representation of a material. In this word we collectively found the "wrong" to be corrected. We had our culprit.
Well all of this to me, back then, sounded a bit like a personal attack. I mean, gradients and shadows was all I did, and just because some were abusing it I had to pay for it? Yeap I did!

Plus Oxygen was starting to look old, in my eyes, and I knew it was time for something new, something fresh. 
The true testament to Oxygen's relevance was about to be put to a test. Would it be replaced, would some breeze of freshness be able to correct all wrongs in Oxygen?
To be absolutely honest, I was worried. For some time it seemed no-one would pick up the work, and from a personal point of view I felt I needed to take some timeout to reinvent myself, so I should not be leading a post-Oxygen design language.

But the magic of open source did offer us, collectively, a new set of fresh people in the form of the KDE Visual Design Group, and with that Breeze, a new past of something new that was its future.

Oxygen will still live on the 5.X's series of Plasma/KDE desktops and I will keep on maintaining it, but now there is a new being that Oxygen failed to be. And it is great.
This was achieved by far more than just me, it was/is a wonderful group of people - some of them I mention above, and trying to mention them all is impossible; but being incredibly unfair to all of the ones I will not mention, a word to Riccardo, Marco, Sho, Bettio, Ken, and all of the users: an enormous Thank You. 

\section*{Conclusion}
So when starting this article the point I wanted to make was the transient nature of design in open source projects. Its announced death at inception time is not something to be taken as a bad thing. It is a natural event that should be embraced from the beginning, the fact that it will have three tempos and that they will cross within themselves is a natural thing, on its way to reach conclusion, that its mortality is nothing but a step into a new birth.
I wanted to say that I knew this from the beginning and that it made me happy that everything went as planned. It did and it is true that I am sincerely happy about the little apparent loop of creation, the cheating of death by continuity, a glimpse of immortality, via the ever chaotic butterfly effect. And this would make a nice enough conclusion advising you, the designers, to embrace open source projects with that in mind.

But... Thinking a bit more about it, I have to be more honest, I have to look into what drove me, what motivated me. I mean having a plan and executing it when you previously define that very plan according to a pattern that you see as the only possible best outcome may feel a bit mechanical, and it was nothing like that.

... and then I remember, people used to ask me all the time “how come you do it?” and I would answer “because it's fun”, this simple answer was my real truth, maybe this is all it boils down to, that terrible cliché. It's not the destination it's the journey, the journey is what makes life and at the end of the day life is only true if you do, see, feel, create, quit, restart, win, lose, and love, yes love. Love what you do! I Loved doing Oxygen. I love doing open source design.
