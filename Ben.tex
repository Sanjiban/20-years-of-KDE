\chapterwithauthor{Ben Cooksley}{Future journeys: which path to take?} 

\authorbio{Ben Cooksley has been a user of free software since 2005, when he stumbled across a Knoppix CD. Since then he has gone on to wear many hats in the KDE project including user support, development, and more recently infrastructure administrator. When not working on KDE he can be found in places he previously hasn't visited or on nature trails.}

\noindent{}As being a well established and successful community, KDE has accomplished a great deal by producing software for a wide variety of roles while having a good deal of fun along the way (although some hair may have been lost, depending on who you ask). We have produced countless different desktops and applications that users love, something proven by the wide variety of user customization and other content that exists for them. But we are not finished yet. The next evolution in technology, and along with it users' devices, awaits us.

Being ready for this evolution is crucial. Countless examples of titans not being ready for change exist: IBM underestimating Microsoft and Intel in the PC revolution; both Microsoft and Nokia failing to ride the smartphone revolution, all despite their former successes. With the rise of smartphones and their attached app stores, our traditional base of both users and new contributors is changing as well and will continue to do so as the next evolution arrives. We must change with it, bringing our software to new places and creating better software others haven't yet thought of.

If you think back to how you got involved in free software, chances are it started with you using the software in some form. Whether it was a bug that bit you, a feature you missed, documentation that lacked answers to your questions or something else, the road to involvement in free software communities such as KDE starts with becoming a user of the software it produces. It certainly did for me. Without a presence on these platforms, people cannot become even aware of our software, let alone try it and become a devoted user. With today's users being tomorrow's contributors, we cannot afford to miss new platforms  as contributors are the lifeblood of not only our, but many other, free software communities.

The creation of better software sounds like a daunting task, like climbing a mountain with no prior experience of doing so. The truth is, the best software is the software that meets the needs of our particular use case the best and presents it in the most accessible form possible. This requires knowing the needs of the particular group of users, what issues they hit, the features they miss, and the things which get in their way. From there, we can set about solving these problems, climbing the mountain as it were, and creating a community of enthusiastic users for whom our software is best in class. In the long term, these very same users will not only spread our software but some will also become contributors, joining the communities that produced the software.

While the path forward may not be entirely known yet, the next 20 years hold many things in store for KDE. Our future software will run on devices we have yet to conceive of and will do things for our users that have yet to even be dreamed of. Yet one thing will remain the same – the creation of software people love – that will inspire the next group of contributors to our community.
