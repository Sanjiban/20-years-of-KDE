\chapterwithauthor{Scarlett Clark}{My Journey From Documentation to Continuous Integration}

\authorbio{Scarlett Clark is \todo{add bio}}

\noindent{}My journey started over three years ago with an email to the KDE documentation list asking a simple question, “How can I help?”. Little did I know the adventures that would transpire from that email! I was quickly pointed to the KDE PIM team to assist them with some documentation.  My first task was a massive rewrite of the KMail documentation, as it was very dated and in the worst shape. This progressed for some time, until I had to step down to resolve some health issues. Upon my return, I had briefly discussed with Valorie and Jonathan about helping out Kubuntu with some documentation as well. Thus begins a new adventure.

The next leg of my journey was Kubuntu. I was working on a bit of Wiki documentation, when Jonathan asked in IRC if anyone was interested in learning packaging. Me being the curious sort that I am, raised my virtual hand. This began a rather long journey into software packaging. It is an exciting facet of software, because you need to make everything work together and it is not always as straight forward as one would think. After some time of my packaging adventure, I was invited to my first Akademy! This was an amazing experience for someone rather new to Open Source contributions. It not only fueled me to further myself into contributing more, it got me closer to the people I have been working with online. It put faces to the IRC nicks! This I feel made a big difference in future online communications, as now personalities are known. Text does not really do justice to personalities and can be quite misunderstood. So now armed with a pile of new friends, I begin my next adventure into Season of KDE.

I had no knowledge of Season of KDE until Valorie had asked me if I would be interested in DevOps type work. She then linked the Season of KDE project to me, my thirst for knowledge triggered, thus beginning the next leg of my adventure. After getting approved for the project by Ben, I started on the revamp of build.kde.org. It began as quite a large learning curve, but as the pieces came together, it got easier. I successfully completed my Season of KDE project and launched my new build.kde.org automating job creation with groovy DSL.  Seeing your creation go live for public consumption is an amazing feeling! With my packaging work and Season of KDE project, I was once again invited to Akademy. This Akademy was very special as I won an Akademy award\footnote{award given out annually at Akademy to KDE contributors}! The jury award to be exact, it was truly exciting to be recognized for all of the work I had put into the CI. The CI system is still a continuing work in progress. We have since moved to using docker containers for our builds. I have recently recruited some help to push forward the other platforms (Windows, Android, and OS X). With new knowledge and experience under my belt, I have rewritten the DSL once again to be much cleaner and extendable. So look forward to much more in the land of build.kde.org!
