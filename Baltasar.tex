\chapterwithauthor{Baltasar Ortega}{A user in the court of KDE developers}

\authorbio{Baltasar Ortega is \todo{add bio}}

\noindent{}This is the personal story of how a simple user has become a member of the KDE community and its daily struggle for the success of a project you're passionate about.

It is written to commemorate the 20 years of the KDE project which I consider extremely important for the development of the software world and where I feel completely integrated but sometimes do not understand the language of the developers with whom I relate almost all days sometimes making me feel like an astronaut at the court of King Arthur.

It all started when, after many unsuccessful attempts and thanks to the Comunidades de Software Libre forums, I could properly configure a USB router in my newly installed GNU/Linux distribution. I think it was a openSUSE 10 and came with the KDE desktop. The idea simply give my PC software that I thought better, something purely selfish.\todo{have someone properly translate this sentence from Spanish}

At that time I did not completely understand the ecosystem of applications but I quickly realized that the KDE project gave me what I wanted: a nice, configurable desktop, free from malware, 100\% translated into my language and with large applications like Konqueror, Kontact Kate, Amarok, digikam, etc.

However KDE Software was just something that was "created out of nothing" and that I used without thinking much about its origin.

But gradually, I was falling in love and the time came when I thought that everyone should enjoy their computer as much as I was.
Thus I decided to help with spreading the word about GNU/Linux and the KDE project but did not know how. I did not know how to program, I could not draw, I did not like to translate and did not know how to package software.

I, as a teacher, knew only how to explain things ... so I created a blog about KDE, where I try to help others get started in this wonderful world.

Thanks to the blog I started learning a lot about the world of Free Software making many mistakes along the way but always learning.
And then I discovered that events were held regularly and discovered that there was a regional group for KDE: KDE Espa\~na. I wanted to become a member and I was surprised that I was accepted. I wanted to attend its annual meeting (Akademy-es 2010 in Bilbao). That meant traveling 600 km from my city to an event where I knew no one except for a few email exchanges. It was the an experience that changed my view of Free Software.

They welcomed me as one of them and I discovered what KDE really is about. There I discovered that behind each application, each translation, each design on my computer there is a person who made it possible. There I discovered that these people have great ethical and moral values ​​and there I discovered I wanted to be a part of this gear.

Suddenly it was no longer only KDE Software, KDE was a project of people making software for other people, respecting their freedoms and privacy. And that was a great desubrimiento\todo{translation} for me, I wanted to be an active part of the KDE community and help spread it.

I understood that the KDE community consists of all kinds of people, not only developers and at first we might think, and myself,\todo{check translation} a user without coding skills, could play an important role in the development of Free Software.

The rest, as they say, is history.

Since that year I became "a user in the court of KDE Developers" working every day to promote free software from my particular point of view, either on the blog, in social networks, giving talks, putting stickers on my computer, teaching my students about free alternatives, working on projects such as Wikipedia or OpenStreetMaps, installing GNU/Linux on computers of friends, organizing events such as the 15th anniversary of KDE or the Jornadas Libres de Vila­real - all without knowing how to write a single line of code.

All this has given me great pleasure when reading a comment saying thank you, it has allowed me to travel, learn languages, learn new working methods and understanding that I am part of a community that works for the good of humanity. But above all it has given me the opportunity to meet extraordinary people who devote so much time to creating a better world.

Finally, I want to thank all the people who have contributed, contribute and will contribute with all their work and their efforts to keep the KDE community active and thriving. It will always give you back more than you contribute.
