\chapterwithauthor{Franklin Weng}{Title}

\authorbio{Franklin Weng is open source developer, translator and promoter from Taiwan. He is the coordinator of KDE's zh_TW translation team since 2006 and the president of the Software Liberty Association Taiwan.}

\todo[inline]{add title}
\todo[inline]{still needs first review}

\noindent{}I'm a maverick. I always am. Well, okay, to some extent.

When I graduated from graduate school in 1999, my classmates were
rushing into the Hsinchu Science Park to get a high-pay job in TSMC
or UMC. I wasn't. I selected a small company outside the Hsinchu Science
Park. That company was developing their own search engine that time,
and I was their 19th employee.

When my colleagues were using Windows 98 and NT 3.0, I used Red Hat
7.2 with Chinese Linux Extension, and KDE/Gnome as my desktop system.
At that time I wasn't bound to a specific desktop system. I started
with KDE, then jumped to Gnome due to the beauty of Gaelon. At that
time, no matter what desktop system I used, I kept finding an useful
mail client.

The first mail client I was satisfied and started translating was
Sylpheed, which was developed by a Japanese developer and based on
gtk. It was good, however at that time it couldn't dock into my system
tray in Gnome. I used Gnome and Sylpheed for some years, then I started
to try KDE again due to KMail. Yes, just because KMail could dock
in the system tray in KDE. It was that simple, and since then I was
bound to KDE till now.

KDE was beautiful. KDE had abundant software for all kinds of jobs
to use. KDE had many cool effects. I loved it, and I started to translate
it, starting from KMail, to many other applications. I was proud to
be the only one in my company who fully used Linux and KDE as my daily
work environment. I was also proud to contribute to the traditional
Chinese translations of KDE, even though I was the only one to do
this.

Then KDE advanced to 4.0. I hadn't changed my desktop to 4.0 yet,
but I had introduced FOSS and KDE 4.0 in some speech that time. It
was buggy, but still gave me a lot of fun. When KDE 4.1 was out, I
fell in love with it again and changed my daily work environment to
KDE 4.1 almost immediately. I still remember the moment I understood
the concept of Plasma. It was when I successfully put the application
menu on the desktop, instead of the panel. It looked a bit silly,
but I was quite excited when I understood the concept.

In the year of 2012 I changed the desktop system from Gnome to KDE
Plasma 4 in ezgo, which was a derived Linux distribution used to promote
FOSS in Taiwan's school. The main reason to change to KDE Plasma was
actually because it kept the application menu while Gnome and Unity
getting rid of it. We had some (well, a lot) arguments with this change,
but we decided to use KDE Plasma 4 finally. In the year 2013 we successfully
\char`\"{}defeated\char`\"{} Microsoft and \char`\"{}resided\char`\"{}
the Linux and KDE Plasma 4 into the 10,000 computers New Taipei City
purchased that year. I also developed some debian packages so that
we could easily change Kubuntu into ezgo. 

Now KDE is no longer a mere desktop system anymore. It is a community.
It is \char`\"{}an international technology team dedicated to creating
a free and user-friendly computing experience, offering an advanced
graphical desktop, a wide variety of applications for communication,
work, education and entertainment and a platform to easily build new
applications upon.\char`\"{}%
\footnote{https://www.kde.org/%
} Just that, I have no idea about how many people being aware of this
change, from a desktop system only to a set of people dedicated to
creating beautiful experiences in the world of FOSS. Maybe, most people
seem to still treat KDE as a desktop system only. Maybe. I just have
no idea about how much efforts were paid on the marketing of this
change. 

KDE is facing a strict trial now. After the iPad was born, it almost
flipped the whole digital world. The famous vision \char`\"{}There's
a computer on every desk\char`\"{} is almost realized now. Facing
such a huge change, what's the future of KDE?

I see that many old communities like Gnome or Mozilla are facing the
same challenge now. We all need to change, no doubt. IMHO, KDE must
keep its characteristic while adapting into the current new digital
world. KDE should use its advantages of being based on Qt, and hence
easily being ported to Android and iOS. KDE has many good applications,
which can be used on Android or iOS with some user interface change.
At the same time, it should not be too hard to keep good performance
on the \char`\"{}old\char`\"{} platform like PC and laptops. Just
that, the marketing part of KDE can and ought to be more compact and
effective.

I'm always proud to be a member of KDE. Let's make it better.

Happy birthday, KDE.