\chapterwithauthor{Franklin Weng}{Title}

\authorbio{Franklin Weng is open source developer, translator and promoter from Taiwan. He is the coordinator of KDE's zh\_TW translation team since 2006 and the president of the Software Liberty Association Taiwan.}

\todo[inline]{add title}

\noindent{}I'm a maverick. I've always been one. Well, okay, to some extent.

When I completed graduate school in 1999, my classmates
rushed to the Hsinchu Science Park to find a high-pay job at TSMC
or UMC. I didn't. I went to a small company outside the Hsinchu Science
Park. That company was developing their own search engine at that time,
and I was their 19th employee.

While my colleagues were using Windows 98 and NT 3.0, I was using Red Hat
7.2 with Chinese Linux Extension, and KDE/GNOME as my desktop system.
At that time, I wasn't bound to a specific desktop system. I started
with KDE, then jumped to GNOME due to the beauty of Gaelon. No matter what 
desktop system I used, I kept looking for a useful mail client.

The first mail client I was satisfied with and started translating was
Sylpheed, which was developed by a Japanese developer and based on
GTK. Sylpheed was good, but at that time it couldn't dock into my system
tray in GNOME. I used GNOME and Sylpheed for some years, but eventually 
tried KDE again for KMail because it could dock in the system tray in KDE. 
It was that simple, and since then I have been bound to KDE.

My first impression was that KDE was beautiful. KDE had abundant software for all kinds of jobs
to use. KDE had many cool effects. I loved it, and I started to translate
for it, starting with KMail, and then contributing to many other applications. I was proud to
be the only one in my company who fully used Linux and KDE as my daily
work environment. I was also proud to contribute to the traditional
Chinese translations of KDE, even though I was the only one to do
this.

Then KDE 4 was released. I had not yet changed my desktop to 4.0,
but I had introduced FOSS and KDE 4.0 in some talk I gave around that time. KDE 4
was buggy, but it still gave me a lot of fun. When KDE 4.1 was released, I
fell in love with KDE all over again and changed my daily work environment to
KDE 4.1 almost immediately. I still remember the moment I understood
the concept of Plasma. It was when I successfully put the application
menu on the desktop, instead of the panel. It is a bit silly,
but I was quite excited when I realized this.

In 2012 I successfully changed the default desktop environment in EzGo to
Plasma 4. EzGo is a derived Linux distribution used to promote
FOSS in Taiwan's school. The main reason to change to Plasma was
because it still had an application menu while the other desktop environments
had removed theirs. We had some (well, a lot of) arguments about this change,
but we finally decided to use Plasma 4. Then, in 2013 we successfully
"defeated" Microsoft and managed to install Linux and Plasma 4 on the 
10,000 computers New Taipei City purchased that year. 
I also created some Debian packages so that we could easily change Kubuntu into EzGo. 

KDE is no longer a mere desktop system but a community of people
dedicated to creating a free and user-friendly computing experience, offering an advanced
graphical desktop, a wide variety of applications for communication,
work, education and entertainment and a platform to easily build new
applications upon. However, I have no idea how many people are aware of the
change from a desktop-centric system to a set of people dedicated to
creating beautiful experiences in the world of FOSS. Maybe most people
seem to still treat KDE as a desktop system only. Maybe. 

KDE is facing a crisis of identity. After the iPad was born, the 
computing world was flipped on its head. The famous vision for 
"a computer on every desk" is almost realized. Facing
such a huge change, what is in the future for KDE?
I see that many old communities like GNOME or Mozilla are facing these
same challenges. We all need to change, no doubt. IMHO, KDE must
keep its identity while adapting into the current new digital
world. KDE should use its advantages of being based on Qt and aim for
being ported to Android and iOS. KDE has many good applications that
could be used on Android or iOS with some user interface changes.
At the same time, it should not be too difficult to keep maintaining software
on the older platforms like PC and laptops. Even so, 
the marketing part of KDE can and should aim to be more compact and
effective.

I'm always proud to be a member of KDE. Let's make it better.
