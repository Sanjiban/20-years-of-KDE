\chapterwithauthor{Cornelius Schumacher}{German Association Law as Secret Superpower}

\authorbio{Cornelius Schumacher became a contributor to KDE in 1999. He maintained KOrganizer, co-founded Kontact, wrote KConfig XT and a lot of other code. In addition to coding, Cornelius served on KDE e.V.'s board for nine years, five years of them as its president. Cornelius works as an engineering manager at SUSE Linux. He holds a degree in physics from the university of Heidelberg.}

\noindent{}When KDE was founded in 1996 it was off to a quick start. It only took days to find a group of core developers, a few weeks to release the first code, and it quickly gained popularity a couple of months after its inception. In 1997 the founders decided to create a non-profit organization incorporated according to German association law in order to support and represent KDE. This decision turned out to be of remarkable foresight.

When people ask if they should found an organization to support their free software project, the usual advice is: "Don't do it". It comes with a lot of work, with headaches of an unusual kind, and requires quite a bit of stamina to follow through. It's worth it, but it's a heavy investment, which only bears fruits if you can sustain it. Nowadays there are a number of organizations which can act as umbrella for free software projects, and many projects actually have gone through the process of creating their own organization. This was different in 1997. KDE was a pioneer.

The Linux-Kongress took place in Würzburg, Germany, in the spring of 1997. It was one of the primary Linux and free software related events at that time. KDE presented its desktop there for one of the first times. This was an overwhelming success. Two members of another desktop project canceled their presentation and left the event after they had seen Matthias Ettrich and Martin Konold show what KDE could already do. Haavard Nord, one of the founders of Trolltech, presented Qt, the toolkit KDE based its software on. He had conversations with the KDE people, which should shape the future of the project. On the way back home to Tübingen the foundation for creating KDE e.V. was laid and Martin then wrote the first version of the articles of association of what would become KDE e.V.

The founding meeting was scheduled for October 1997 in the student apartment of Matthias Ettrich. German law requires seven people to physically be present as initial members to found an association. As KDE already was a distributed international project back then, they had to recruit Matthias' room mates to meet the required number. The initial members voted about the articles and the association was born. Matthias became its first president, and Kalle Dalheimer and Martin its vice presidents. That out of the way, it's said that the group then quickly turned to technical matters again and established a packet radio connection to the neighbor's apartment.

One of the primary reasons for creating KDE e.V. was the need to handle money for developer meetings. The first KDE meeting in Arnsberg in September 1997 already had a budget of a few thousand Euros, and handling this in a private way was getting impractical. With KDE e.V. there was the right mechanism in place to handle that more professionally, and it turned out to be one of the most powerful catalysts for the community. Hundreds of developer meetings followed and dozens of conferences all over the world. Without KDE e.V. as secret superpower behind it this wouldn't have been possible.

The first critical test for KDE e.V. came in 1998. There was a debate about the license of Qt. Some people held the opinion that it wasn't free enough. Trolltech, the owner of Qt, decided to do something about it and entered into an agreement with KDE that would guarantee the availability of Qt under a free license. The KDE Free Qt Foundation was born, founded by Trolltech and KDE e.V. Matthias and Kalle flew to Oslo to sign the contracts and established a cornerstone of Trolltech's commitment to free software and KDE.

The agreement of the KDE Free Qt Foundation is a defensive agreement, which comes into effect in the case that Qt wouldn't be released under a free software license anymore. While it hopefully never needs to be exercised it gives KDE a seat at the table when Qt licensing is discussed. In 2008 this suddenly became very important when Nokia announced to buy Trolltech including Qt.

Nokia acted as an exemplary free software citizen. They reached out to the KDE e.V. board as one of the first things after the acquisition. Nokia managers came to Frankfurt, where KDE e.V. had established its first office together with Wikimedia Germany, for a remarkable meeting, and KDE e.V.'s board returned the favor by going to Nokia's headquarter in Helsinki to discuss collaboration and strategy. It was a fruitful time with lots of investment in Qt and KDE as well. It turned out that KDE's strategy outlived Nokia's in the end, though.

The meetings, conferences, and the agreement about Qt are only a few examples of how KDE e.V. backed the community in a way which wouldn't have been possible without an organization such as this. KDE always has been a playful bunch of people, who focused on technology. There are countless examples of brilliant technological innovation in KDE's history. But to operate a community on the scale of KDE did require more than that. There were three domains, which were of special significance here: Representation, support, and providing governance.

When a group of volunteers comes together to work on free software, the means of how technical collaboration happens fall into place quite easily. Tools such as git, mailing lists, or IRC provide the distributed infrastructure to develop and discuss code. The philosophy of free software and its development processes provide a solid base for decision making and coordinating work. Two cornerstones of KDE's philosophy always were the common ownership of code, and the mantra that those who do the work decide.

Sometimes it needs more than that. Entering formal agreements needs an entity which can act on behalf of the community. The KDE Free Qt Foundation is an example from the early days of KDE which is still highly relevant. There are more examples, such as holding the registration of the KDE trademark, owning the kde.org domain, being partner in EU research projects, receiving money on behalf of the community, maintaining copyright through the fiduciary license agreement created by the Free Software Foundation Europe, dealing with legal obligations such as taxes, or simply having a central point of contact for people who need an entry point to the community. This is where KDE e.V. represents the community.

Millions of Euros have gone into the community through KDE e.V. over the years. Lots of individuals support KDE by donations. Companies give back by providing financial support. KDE participates in programs where some money goes to the organization for work and support it provides. KDE has been a partner in Google's Summer of Code in all the twelve years of its existence, and KDE e.V. handles the money which goes into supporting the KDE mentors in the program. This money goes back to the community, helping people to attend conferences or developer meetings, providing technical infrastructure, paying for necessary administrative efforts. This is where KDE e.V. supports the community.

Finally KDE e.V. provides governance to the community. This actually comes with a twist, because KDE e.V. does not control or steer the technical development of KDE software. This was one of the conscious decisions when setting up the organization. The open source development process provides culture and practices to take decisions and run development based on the motivation and responsibility of the individual contributors. This is a powerful concept, which doesn't require a central authority to plan and control actions. It can even be harmful to try to exercise central control in that context. So KDE e.V. decided to stay out of this domain.

KDE e.V. does not stay out of the question who ultimately owns KDE, and who manages its assets. The German association law provides a strong and solid base here. It is used by hundreds of thousands of associations in Germany, and it provides a democratic and transparent base of how to run an organization for the public benefit. This makes sure that KDE can't be bought or hijacked. It always will be in the hand of its constituency, the people who put in their time and effort as contributors. So KDE e.V. is the body who is set up to give legitimacy to efforts within the community and to people who act on behalf of it. This is where KDE e.V. provides governance to the community.

Today KDE e.V. is a well oiled machine, which represents, supports, and provides governance to the community. There is an annual general assembly where its members meet to report about work being done, elect the board and other representatives of KDE e.V., and vote about the main decisions taken by the organization. There are regular reports and discussions about ongoing topics, and there is a team taking care of the daily business. This wasn't always a given, though. There was a critical phase in KDE e.V.'s development, where the sustainability of the organization was threatened.

In the early years of the millennium after the tech bubble imploded, there was less money available for technology. Many people changed projects, some people who were paid to work full-time on free software moved on to other endeavors. This also affected KDE. While the development was in full swing driven by enthusiastic volunteers, the organizational side starved. KDE e.V. was in limbo, because too many active members had gone missing.

In 2002 Mirko Boehm, treasurer of KDE e.V at that time, organized a memorable meeting. He invited the members of KDE e.V. and most of the active core contributors of KDE to a general assembly. The goal was reviving the organization, putting an active board in place, and revising the articles of association to deal with members becoming non-active and setting up the foundation for being officially accepted as tax-exempt non-profit organization for the public benefit.

The meeting took place in Hamburg, hosted by the university of the German armed forces. There were meeting rooms, a computer lab, which was quickly hacked to provide the means to compile KDE software, and dormitories for the short times when people needed to keep up with sleep. Rumors say that they were torn down directly after the meeting.

It took two more years to implement all the decisions but with the changes of the articles of association it defined the modern KDE e.V. in the way it still is in place today. A new board was elected bringing in Eva Brucherseifer who would run the organization for the coming five years. The meeting also facilitated a discussion about KDE's values triggered by a document a young ambitious developer from Canada, Aaron Seigo, had written. Aaron couldn't be present at the meeting in Hamburg, but he had a role in the future of KDE e.V., serving as its president from 2007 to 2009.

KDE e.V. went on a growing trajectory over the subsequent years after the meeting. It was able to provide support to the community again, opened an office, employed Claudia Rauch as a business manager in 2008, and dealt with a lot of the work behind the scenes to run the community, to organize its events, such as the memorable Desktop Summits in Gran Canaria and Berlin in 2009 and 2011, hundreds of developer meetings, and many more.

In 2010 it even signed a Hollywood contract. The production of Robert Rodriguez' Machete wanted to include pictures of KDE software in the movie. So KDE e.V. as owner of the trademark and representative of the community jumped in to sign the agreement and allow Robert de Niro and Lindsay Lohan to communicate via KDE's instant messenger.

It might seem surprising, but the solid base of the German association law, the foresight of the founders to create an organization based on this law, and the relentless work of countless people to run this organization did create a secret superpower for the KDE community. May the community continue to use it wisely. There is a lot of good to be done.
