\chapterwithauthor{Aracele Torres}{KDE ceased to be a software and has become a culture} 

\authorbio{Aracele Torres is TODO}

\noindent{}KDE as a software project born in 1996, when Matthias Ettrich, a German programmer, realized that Unix-based systems were growing, but their interfaces were not friendly enough to the end user. There were only three years that the first GNU/Linux distributions had begun to appear and Matthias noticed the absence of a GUI (Graphical User Interface) that offered a complete environment for the end user to perform its daily tasks. He thought then for the people began to consider the GNU/Linux as an alternative to proprietary systems, a beautiful and easy to use graphical environment would help a lot. Thus was born the project "Kool Desktop Environment" or simply "KDE". The name was a pun on the proprietary graphic environment very popular at the time, CDE (Commond Desktop Environment), which also ran on Unix systems.

One year after Matthias Ettrich notice inviting developers to join your project, the first beta of KDE was released. Nine months later, the first stable release. It was the dream of Matthias and its community of contributors becoming real and occupying an important place in the history of free software. The project then was maturing and becoming more complex and more complete. Thanks to the collaboration of people worldwide, KDE has grown from version 1 to 2, in 2000, then to 3 in 2002. In 2008, after very important changes, the community launches revolutionary version 4. In 2014, the equally innovative version 5, which reveals in its visual and its frameworks structure and applications that the community is ready for the future.

During that time a lot has changed in the community in addition to its technologies. There was a change in its own name and its own identity. In 2008, the community began to refer to "KDE" no more like a software project, but as a "global community." This identity change was made official in 2009, when the community announced its rebranding. At the time, it was announced that the name "K Desktop Environment" would be abandoned because it no longer represented what the KDE had become. This long name had become obsolete and ambiguous, since it represented the desktop that the community has developed. At that time "KDE" was not just a desktop, it was something bigger, as the announcement said: “KDE is no longer software created by people, but people who create software”. 

Many may not have noticed, but this rebranding was a turning point in the KDE history. In it the community makes clear its tendency to perceive and keep up with state of the art of computing. The reign of desktops was over and it made no sense to limit it. Therefore, the decision to use only the name "KDE" intended to communicate to users the message that the community was attentive to the future. "KDE" would no longer be synonymous with a limited set of software components, but an international community that produces free technologies to the end user, whether for desktop or mobile devices or other technologies that are yet to come. 

In 2012, this process of change was synthesized in a manifesto. The "KDE Manifesto" listed the core values advocated by the community and it worked also as a kind of warning: the community was open to house new projects in its umbrella brand. For it was started in 2014 KDE Incubator, an incubator of projects that join the community and have the same benefits as other native projects. This incubator is now home to varied projects, from a wiki dedicated to educational topics to a distro. 

So the community continues its trend of the past 20 years, which is inclusive growth. In the past it grown to provide its users a better desktop experience. At present it continues expanding and changing also to offer the best experience on mobile devices. In the future it is very likely that it starts to create products for a new generation of devices: cars, smart TVs, refrigerators, stoves, etc. It is possible that smart homes, maybe even an entire city can use the technologies of the KDE community to work. Why not? 

For almost 10 years I have used the technologies that the KDE community produces. I remember I started with KDE 3.5, when I still did not even know what the social importance of free software. As a user, as a contributor and as a historian, to look at these twenty years of history, the feeling I have is the same. The KDE project was born as the concern of a group of people make computing, especially free and open computing, accessible to all. In the 1990s, when it was created, it was a time when the GNU/Linux systems were becoming popular and the internet was also becoming something more present in people's lives, especially after the Web. KDE then emerges as an important tool in the popularization of free software. It was an interface that helped the man to understand the machine and communicate through it. Today KDE is an interface between people. It is what unites and connects them towards a free computing. It is a community that encourages the growth of people and projects, that seeks innovation, which defends the free sharing of information. KDE ceased to be a software and has become a culture.
