\chapterwithauthor{Aracele Torres}{KDE Ceased to Be Software and Has Became a Culture} 

\authorbio{Aracele Torres fell in love with KDE technologies in 2007 and in 2010 decided that she should start to contribute to the community. Since then she has made contributions in many areas, such as translation, promotion, artwork, and community management. She travels through Brazil, giving talks about KDE and organizing activities to promote the community. Additionally she participates in the organization of KDE's Latin-America Summit LaKademy since its first edition. She is a doctoral student in the History of Science and Technology, conducting research on the history of digital technology, free software, the internet and related things.}

\noindent{}KDE as a software project was born in 1996 when German programmer Matthias Ettrich realized that Unix-based systems were growing, but their interfaces were not user-friendly enough for the end user. It was only three years after the first GNU/Linux distributions had begun to appear and Matthias noticed the absence of a graphical user interface that offered a complete environment for the end user to perform their daily tasks. He thought that for the people who began to consider GNU/Linux as an alternative to proprietary systems, a beautiful and easy to use graphical environment would help a lot. Thus was born the project "Kool Desktop Environment" or simply "KDE". The name was a pun on the proprietary graphical environment very popular at the time, CDE (Common Desktop Environment) that also ran on Unix systems.

One year after Matthias Ettrich's announcement inviting developers to join the project, the first beta of KDE was released. Nine months later came the first stable release. The dream of Matthias and his community of contributors was becoming reality and occupying an important place in the history of free software. The project was maturing and becoming more complex and more complete. Thanks to the collaboration of people worldwide, KDE had grown from version 1 to 2 in 2000, and then to 3 in 2002. In 2008, after very important changes, the community launched the revolutionary version 4. In 2014, the equally innovative version 5 came out that showed in its visual design, framework, and applications that the community is ready for the future.

During this time, a lot had changed in the KDE Community and its technologies. First came the change of KDE's name and its identity. In 2008, the community began to refer to "KDE" as not just a software project, but as a global community. This identity change was made official in 2009 when the community announced this rebranding. The name "K Desktop Environment" was dropped because it no longer represented what KDE had become. This long name had become obsolete and ambiguous, since it represented a desktop that the community has developed. KDE at that point had become more than just a desktop, evolving into something greater: \textit{KDE was no longer the software created by people, but the people who create software.} 

Outside observers of the KDE Community may not have noticed, but this rebranding was a turning point in the history of KDE. Through it, the community makes clear its ability to perceive and keep up with the state of the art of computing. The reign of desktops was over and it made no sense to limit KDE to traditional platforms. Therefore, the decision to use only the name "KDE" intended to communicate to users that the community was attentive to the future. "KDE" would not only be synonymous with a limited set of software components, but be synonymous with the international community that produces free technologies for the end user, whether for desktop or mobile devices or other technologies that are yet to come. 

In 2012, this rebranding effort was summarized in a manifesto. The "KDE Manifesto" listed core values advocated by the community and it served as an open call: New projects are now welcome under the KDE umbrella brand. This led to the creation of the KDE Incubator in 2014. The KDE Incubator serves as an integration point for new free software projects that join the community to have the same benefits as the existing KDE projects. This incubator has become home to a variety of projects, from a wiki dedicated to educational topics to a full-fledged Linux distribution. 

KDE continues the same trend it has followed the past 20 years, which is inclusive growth. Its founding principle was based on providing users a better desktop experience. It has expanded that view to also offer the best experience on mobile devices. The future is ripe with possibilities for a new generation of devices: cars, smart TVs, refrigerators, stoves, etc. Can you imagine a smart home or even an entire city using the technologies of the KDE Community? 

For almost 10 years I have used the technologies that the KDE Community produces. I remember starting with KDE 3.5 when I still did not even know about the social importance of free software. As a user, as a contributor, and as a historian, when I look back at these 20 years of history, the feeling is the same. The KDE project was born from the interest of a group of people who wanted to make computing, especially free and open computing, accessible to all. The 1990s was a time when the GNU/Linux systems became popular and the internet and the web was becoming more present in people's lives. KDE emerged as an important tool in the popularization of free software. Born with the intent to be an interface between a person and the computer, today KDE is an interface between people. KDE unites and connects people through free computing. KDE has become a community that encourages the growth of people and projects, that seeks innovation, and defends the free sharing of information. KDE has moved beyond simply being computer software and has evolved to become a culture.
