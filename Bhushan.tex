\chapterwithauthor{Bhushan Shah}{Story of a contributor}

\authorbio{Bhushan Shah is \todo{add bio}}

\noindent{}Lydia Pintscher asked me while ago if you want to write a essay in a book for KDE's 20th anniversery on the topic of "KDE: past, present and future". At start I thought when KDE was started in 1996, Perhaps I was still learning to speak. And my memory of past two years is still fresh as present moment. What will I write?

\paragraph{Introduction} At a time of writing this I am celebrating the two year anniversery of getting Developer account. I started to contribute to KDE by sending minor patches and eventually I participated in Season of KDE 2013, outreach program organized by KDE. Sebastian Kügler from Plasma team mentored me, this was great experience. After that I enrolled into Google Summer of Code twice as student, this year I am helping to adminstrate the Google Summer of Code and other mentoring programs. On the development side I currently work on Plasma Mobile, free software eco-system for mobile devices.

\paragraph{Mentoring programs} KDE community runs and participates into various mentoring programs, for example Google summer of Code, Google Code-in, Seaon of KDE, Outreah program for woman. As I got into KDE community through one of this program I am currently passing the torch and mentoring other students and helping to adminstrate this programs. One of the reason I feel too good about this is, it gives me feeling of "giving back". KDE as community welcomed me and made me one of them when I was eager to learn and contribute to KDE community. Now I am helping others for same. New contributors helps to create more diverse community and keeps fun in community intact. In my two years of experience with KDE community I've come across various new contributors from various professional and cultural backgrounds. This also helps the naive contributors to build their skills with help from existing contributors of KDE.

When I started contributing to KDE I didn't really had much skills, all I knew was I want to contribute to KDE. At that time Plasma team helped me to learn important aspects of Programming and KDE codebase. First patch that I submitted got more then 20 issues to resolve, at that point I also learned important skill, dealing with feedback effectively. Eventually I learnt more skills and started to contribute to more advanced parts of the KDE Plasma Workspaces. For this I thank my mentors, Sebastian Kügler, Shantanu Tushar, Sinny Kumari, Marco Martin and KDE community as whole.

\paragraph{Plasma Mobile} Plasma Mobile project's first prototype was revealed at Akademy 2015 by Sebastian Kügler and Boudewijn Rempt on behalf of Plasma team. Plasma Mobile project have following vision set,

\begin{quote}
Plasma Mobile aims to become a complete software system for mobile devices. It is designed to give privacy-aware users back the full-control over their information and communication. Plasma Mobile takes a pragmatic approach and is inclusive to 3rd party software, allowing the user to choose which applications and services to use. It provides a seamless experience across multiple devices. Plasma Mobile implements open standards and it is developed in a transparent process that is open for the community to participate in.
\end{quote}

\noindent{}Which in my opinion plays really well with the vision of the KDE,

\begin{quote}
A world in which everyone has control over their digital life and enjoys freedom and privacy.
\end{quote}

\noindent{}From start I was too excited about this project, Despite being new project, KDE community already have experience working with non-desktop devices, in specific, Plasma Active project. Currently my work is to build the flashable images with Plasma Mobile, and I believe Plasma mobile is important contribution towards Free software movement, and I am glad to be part of it.

\paragraph{Conclusion} On closing words, I want to Thank to whole KDE community, every user, developer, artist, translator.. You are the one who makes it possible to dream Konqi ruling the world. ;-)
