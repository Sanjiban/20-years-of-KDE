\chapterwithauthor{Bhushan Shah}{Story of a contributor}

\authorbio{Bhushan Shah is active contributor from India, previously student of Diploma in Information Technology, who started contributing to KDE two years ago. Currently he maintains the flashable images for Plasma Mobile and is also one of the adminstrator for KDE's mentoring programs.}

\noindent{}I am celebrating the two year anniversary of getting my developer account. I started to contribute to KDE by sending minor patches and eventually in 2013 I participated in Season of KDE, the outreach program organized by KDE. Sebastian K\"{u}gler from the Plasma team mentored me. This was a great experience. After that I enrolled into Google Summer of Code twice as a student - this year I am helping to administrate Google Summer of Code and other mentoring programs for KDE. On the development side I currently work on Plasma Mobile, the free software eco-system for mobile devices.

\paragraph{Mentoring programs} The KDE community runs and participates in various mentoring programs, for example Google Summer of Code, Google Code-in, Season of KDE, FOSS Outreach Program for Woman. As I got into KDE community through one of this program I am currently passing the torch and mentoring other students and helping to administrate these programs. One of the reason I feel good about this is, it gives me feeling of "giving back". KDE as community welcomed me and made me one of them when I was eager to learn and contribute to the KDE community. Now I am helping others experience the same. New contributors help to create a more diverse community and keeps the fun in the community intact. In my two years of experience with the KDE community I've come across various new contributors from various professional and cultural backgrounds. This also helps the naive contributors to build their skills with help from existing contributors of KDE.

When I started contributing to KDE I didn't really have a lot of skills. All I knew was I want to contribute to KDE. At that time the Plasma team helped me to learn important aspects of programming and the KDE codebase. The first patch that I submitted had more then 20 issues to resolve. At that point I also learned an important skill: dealing with feedback effectively. Eventually I learned more skills and started to contribute to more advanced parts of the KDE Plasma Workspaces. For this I thank my mentors, Sebastian K\"{u}gler, Shantanu Tushar, Sinny Kumari, Marco Martin and KDE community as whole.

\paragraph{Plasma Mobile} The Plasma Mobile project's first prototype was revealed at Akademy 2015 by Sebastian K\"{u}gler and Boudewijn Rempt on behalf of the Plasma team. The Plasma Mobile project has the following vision:

\begin{quote}
Plasma Mobile aims to become a complete software system for mobile devices. It is designed to give privacy-aware users back the full-control over their information and communication. Plasma Mobile takes a pragmatic approach and is inclusive to 3rd party software, allowing the user to choose which applications and services to use. It provides a seamless experience across multiple devices. Plasma Mobile implements open standards and it is developed in a transparent process that is open for the community to participate in.
\end{quote}

\noindent{}Which in my opinion plays really well with the vision of KDE,

\begin{quote}
A world in which everyone has control over their digital life and enjoys freedom and privacy.
\end{quote}

\noindent{}From the start I was very excited about this project. Despite being a new project, the KDE community already has experience working with non-desktop devices through the Plasma Active project. Currently I am working on building flashable images with Plasma Mobile. I believe Plasma Mobile is an important contribution to the Free Software movement, and I am glad to be a part of it.
\linebreak 
\linebreak
I want to thank the whole KDE community, every user, developer, artist, translator.. You are the one who makes it possible to dream of Konqi ruling the world. ;-)
