\chapterwithauthor{Yash Shah}{A Revolution in Itself}

\authorbio{Yash Shah is an active contributor to KDE since 2011. He started his journey with contributing to the speech recognition program Simon through Google Summer of Code
 and then along with the KDE India team, he organized two of the largest KDE conferences in India - KDE Meetup 2013 and conf.kde.in 2014. He loves evangelizing KDE and motivating students to contribute to open source in different colleges in India.}

\noindent{}One of the largest international open source conferences was organized in India: The KDE Meetup 2013 and conf.kde.in 2014 at Dhirubhai Ambani Institute of Information and Communication Technology (DA-IICT) in Gandhinagar, Gujarat. It was a platform for the exchange of ideas and thoughts with speakers coming from different cultures, leading to an advancement in the essence of open source development and reliving the joys of contributing in KDE. Both the events were a huge phenomenon in itself with the participation of over 350 students from the far ends of the country such as Delhi, Durgapur, Mumbai and many more. 

It was always my dream to host something at a scale like this in Gujarat. People in our state were not aware of FOSS. With over 150 engineering colleges, there was a huge opportunity to introduce them to the world of contributing to FOSS. When my colleagues and I first started planning the event, we never imagined that we would receive so much support from across India and from around the world. We never thought in our wildest dreams that we might attract an amazing group of developers and open source enthusiasts from more than 9 states in India. Both conferences served for a perfect environment for getting people to know about KDE and open source software development in general. The expert KDE contributors flew in from different parts of India, Europe and USA to talk about KDE applications, introducing the audience to KDE and open source tools and technologies and answer their questions.

One thing that we experienced is that international conferences such as these serve as the perfect arena for students to get involved with open source and to get themselves acquainted with the way communities such as KDE function. These events are the perfect opportunity to interact with these mentors and members who can guide them along and also help them to participate in various coding mentoring programs such as Google Summer of Code and Season of KDE. Students get to collaborate with them and also build upon their own ideas and create new projects of their own as a part of KDE.

One of the major highlights of the event - the Bardoli incident - is an inspiration and an indication of the dedication of the members of the KDE Community to spread knowledge among all and to ensure that everyone is a part of  the community and that no one feels left out. Many enthusiastic students from Bardoli sacrificed their weekend time off and came from far away places and were prepared to make the most of the event. Despite the long journey accompanied by fatigue, they did their best to work along with the speakers but somehow it did not work out for them and they decided to leave the event. Pradeepto saw them leaving and he sat with all of them and talked personally to each one of them. He convinced them to stay. The next day, special sessions were organized just for them and all of them showed up with renewed interest. They covered all that they had missed in a short span of time and were eager to learn more.

Let me share an insight to the positive things people consumed from the conference and how the talks changed their way of thinking. This was a text from a delegates of conf.kde.in  who traveled 700km to attend it:

\begin{quote}Nobody cares about the people, only wants gathering and all. No one knows what were participants doing. But as we cared a lot by you and your team is really appreciated. Thanks for my side behalf of all participant.
\newline
\#kdemeetup One thing to learn.. Don’t be scared of coding.\footnote{\url{https://mail.kde.org/pipermail/kde-india/2014-April/001236.html}}\end{quote}

At the end of the event there was a newfound inspiration in all the participants who had the desire to contribute and to be a part of the largest and friendliest community in the world. This was a change, a different sort of exposure which the students received which relieved them from the usual drudgery and boredom of college education which does not expose them to real-world programming. The event lived up to its aim - helping people know about KDE and to provide them with the basic skills and techniques so that they can contribute to open source. The community speakers were as warm and amiable as they could be and the students were encouraged to approach them and ask as many doubts as they could. This event left the students asking for more. All that they had in their minds was to learn and explore and innovate - create something which could be the next revolution. It gives us motivation to hold such kind of conferences at other places so that young students can benefit more than ever and the culture of FOSS keeps thriving in this world.
