\chapterwithauthor{Sandro Andrade}{On Subtle Contributions}

\authorbio{Sandro Andrade has been using Qt and KDE technologies since 2000, when he found out a new tool to create amazing C++ visual applications -- KDevelop. Sandro is one of the leading contributors of actions which have been fostering KDE in Brazil and South America, such as the creation of the Latin-America KDE Summit (LaKademy) in 2010. He was a member of the KDE e.V. Marketing Working Group, is the creator and maintainer of Minuet (the KDE-edu software for music education) and is currently a member of the KDE e.V. Board of Directors. Sandro holds a PhD in Computer Science and works as a professor in the Computer Science Department at IFBA, Brazil.}

\noindent{}All of us want to see our beloved free software project succeed. Indeed, we do a lot of consciously planned work to achieve such a freedom nirvana. We strive to come up with a nice idea, decide about promising technologies to adopt, and code hopelessly to have an initial release to show off and maybe herd some new users and contributors around the world. If you deliver your software as part of a well-established and large community such as KDE, a brave army of translators, designers, testers, sysadmins, packagers, and promo people is promptly put in place to provide you all needed support.

From the end-user perspective, only a small portion of such a huge effort become effectively apparent, in terms of graphical interfaces, user documentation, helpful features, and so on. As you get more involved in the community and start delighting the full FLOSS experience, all those multifaceted contributions from different people start become apparent and you are immediately snatched up with an irrevocable desire of saying a huge "Thank you" to all those contributors. That said, I would like to talk herein about a different sort of contribution though: the subtle contribution.

I started pushing KDE forward in Brazil in 2008 along with Tomaz Canabrava, when we presented a Qt short-course in one of the biggest Brazilian FLOSS conferences. At that time, only three Brazilian contributors were doing some nice but somehow stand-offish work related to development, translation, and packaging. In 2010 -- as a result of many fostering actions we did in 2009 at some universities and local FLOSS events -- we witnessed the dawn of Akademy-BR, the first Brazilian KDE sprint ever and expanded to the Latin-America KDE Summit (LaKademy) two years later, in 2012.

Nowadays, the Latin-American KDE contributor base is formed by 22 volunteers from Argentina, Brazil, Colombia, and Peru; working on activities related to development, translation, promotion, sysadmin, and artwork. After nearly nine years promoting KDE, doing some development and artwork contributions, and lately working on the overall community management, I now wonder what is that essential element that bonds people together, cultivate thriving atmosphere, and makes such an odyssey lasting for 20 years.

People bond each other by affinity. In FLOSS communities, more often than not, they bond by technological affinity, common interests in some application domain, or reciprocal wish to share knowledge. Whilst such coalescence factors are certainly quite important to build solid and healthy communities, I'd like to emphasize the importance of contributions that goes readily unnoticed and, in spite of such contributor's unawareness, play a definite role when forging FLOSS communities. Failing in recognize such 'subtle contributions' not only hampers the catalisis of community growing but also lets 'subtle contributors' unknown of their own importance for the whole FLOSS ecosystem.

The first kind of subtle contribution I want to describe is 'people as extrinsic motivation'. Although I had been using Qt and KDE technologies for seven years already, it was only in 2008 when I and Tomaz had a conversation which ended with a "do you want to do that?", "if you want to do that, I also want to do", "cool, let's do it". For some people, making things alone may become extremely troublesome. Having someone just to say "you can!" or even walk the path alongside with you is one of the most beautiful subtle contributions I ever seen.

Another subtle contribution is made by people able to proactively identify the small things that engender a welcoming and relaxed atmosphere. And that includes everything someone involved with coding, testing, translation, infrastructure, and packaging usually isn't able to identify, let alone execute it. I am writing this book chapter in a hotel's room at Porto Alegre (south Brazil), at the very last day of the 17th International Free Software Forum (FISL), where KDE took up an amazing booth, presented a whole day of nice talks, and celebrated its 20th anniversary. What a nice demonstration of subtle contributions!

Do you know that invigorating thrill and fond memories when a FLOSS meeting or sprint comes to its end? New features, bug fixes, and translations are awesome but I'd risk to say such a nirvana is primarily caused by a different stimulus. Subtle contributors turn us into a family. They care about how to decorate our booth, they come up with a big commemorative cake, they talk about life or they do not even need to talk to let us know they are quite confortable and happy in being part of KDE.

I regret not being sensitive enough to realize KDE is a stronghold of subtle contributors earlier. I would have said more 'Thank you's.
